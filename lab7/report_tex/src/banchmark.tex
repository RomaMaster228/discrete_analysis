\section{Тест производительности}

Убедимся, что построенный алгоритм действительно имеет сложность $O(n * m)$. Для этого замерим время работы программы на нескольких тестах: с площадями исходного прямоугольника 100, 1000, 10000.

\begin{alltt}
roma@DESKTOP-JD58QU2:~/Diskran/lab7$ ./a.out < test1e2
Time: 0.000460 s
roma@DESKTOP-JD58QU2:~/Diskran/lab7$ ./a.out < test1e3
Time: 0.005608 s
roma@DESKTOP-JD58QU2:~/Diskran/lab7$ ./a.out < test1e4
Time: 0.0011082 s
\end{alltt}

Видно, что время работы программы возрастает прямо пропорционально объему входных данных, значит сложность программы действительно равна $O(n * m)$.

\pagebreak

