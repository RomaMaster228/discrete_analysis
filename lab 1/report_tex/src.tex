\section{Описание}
Требуется написать реализацию алгоритма поразрядной сортировки.

Как сказано в \cite{Kormen}: \enquote{Сам алгоритм состоит в последовательной сортировке объектов какой-либо 
устойчивой сортировкой по каждому разряду, в порядке от младшего разряда к старшему, после чего последовательности 
будут расположены в требуемом порядке}. В качестве устойчивой сортировки для разряда использую устойчивую версию 
сортировки подсчётом. 


\pagebreak

\section{Исходный код}

Для хранения пар \enquote{ключ-значение} будем использовать структуру $TPair$, так как это удобно. Ключ будем хранить
в статическом массиве на 32 элемента, так как считываем его по одному символу. А для значения будем выделять динамическую память, так как там могут быть строки переменной длины до 64 символов.

Поразрядная сортировка $RadixSort$ реализуется итерацией по каждому символу ключа элементов массива, начиная справа.
Эти символы сортируются с помощью сортировки подсчётом $CountingSort$, чтобы программа работала за линейное время.
 
Опишем функцию $CountingSort$. Устойчивая сортировка подсчётом осуществляется с помощью двух вспомогательных массивов: 
один - для счётчика, другой - для записи отсортированного результата. Для удобства обозначим: $v$ - исходный массив, 
$res$ - результирующий массив, $count$ - массив для подсчёта вхождений. 

Сначала заполняем $count$ нулями. Затем для каждого $v[i]$ увеличиваем $count[v[i]]$ на единицу. Так, мы посчитаем 
количество вхождений каждого элемента. Суммируем каждый $count[i]$ с $count[i - 1]$, кроме $count[0]$ (насчитываем 
префиксные суммы). 
Далее читаем входной массив с конца, записываем в $res[count[v[i]]]$ $v[i]$ и уменьшаем $count[v[i]]$ на единицу.



\begin{lstlisting}[language=C++]
#include <iostream>
#include <vector>

using namespace std;

struct TPair {
    char Key[32];
    char* Value;
};

istream& operator>> (istream& input, TPair& pair) {
    for (int i = 0; i < 32; ++i) {
        input >> pair.Key[i];
    }
    input >> pair.Value;
    return input;
}

ostream& operator<< (ostream& output, TPair& pair) {
    for (int i = 0; i < 32; ++i) {
        output << pair.Key[i];
    }
    output << '\t' << pair.Value;
    return output;
}

void CountingSort(int i, vector<TPair>& v) {
    vector<TPair> res(v.size());
    int count[16] = { 0 };
    for (long long j = 0; j < v.size(); ++j) {
        if (v[j].Key[i] - '0' - 49 >= 0) {  // ASCII codes: 'a' = 97, '0' = 48
            ++count[v[j].Key[i] - '0' - 39];
        }
        else {
            ++count[v[j].Key[i] - '0'];
        }
    }
    for (int j = 1; j < 16; ++j) {
        count[j] += count[j - 1];
    }
    for (long long j = v.size() - 1; j >= 0; --j) {
        if (v[j].Key[i] - '0' - 49 >= 0) {
            --count[v[j].Key[i] - '0' - 39];
            res[count[v[j].Key[i] - '0' - 39]] = v[j];
        }
        else {
            --count[v[j].Key[i] - '0'];
            res[count[v[j].Key[i] - '0']] = v[j];
        }
    }
    v = move(res);
}

void RadixSort(vector<TPair>& v) {
    for (int i = 31; i >= 0; --i) {
        CountingSort(i, v);
    }
}

int main() {
    ios::sync_with_stdio(false);
    cin.tie(nullptr);
    cout.tie(nullptr);
    vector<TPair> v;
    TPair pair;
    pair.Value = (char*)malloc(sizeof(char) * 64);
    while (cin >> pair) {
        v.push_back(pair);
        pair.Value = (char*)malloc(sizeof(char) * 64);
    }
    if (v.size() == 0) {
        return 0;
    }
    RadixSort(v);
    for (int i = 0; i < v.size(); ++i) {
        cout << v[i] << '\n';
    }
    return 0;
}

	
\end{lstlisting}

\pagebreak

\section{Консоль}
\begin{alltt}
roma@DESKTOP-JD58QU2:~/Diskran/lab1$ g++ lab1.cpp
roma@DESKTOP-JD58QU2:~/Diskran/lab1$ cat test1 
00000000000000000000000000000010	hello
00000000000000000000000000000001	hi
00000000000000000000000000010010	are
00000000000000000000000000000100	what
roma@DESKTOP-JD58QU2:~/Diskran/lab1$ ./a.out < test1 
00000000000000000000000000000001	hi
00000000000000000000000000000010	hello
00000000000000000000000000000100	what
00000000000000000000000000010010	are
\end{alltt}
\pagebreak

