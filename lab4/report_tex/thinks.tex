\section{Выводы}
{\itshape }

Выполнив четвёртую лабораторную работу по курсу \enquote{Дискретный анализ}, я научился реализовывать алгоритм поиска подстроки в строке Ахо-Корасик. Данный алгоритм обычно сравнивают с суффиксными деревьями. У них одинаковая ассимптотика, но разное потребление памяти, так 
как в алгоритме с суффиксными деревьями мы кладём в бор текст, а не паттерны как в Ахо-Корасик. Следовательно, если текст слишком большой, 
то целесообразнее использовать алгоритм Ахо-Корасик, а если много паттернов и надо давать быстрый ответ, тогда лучше использовать алгоритм
с суффиксными деревьями.
\pagebreak
