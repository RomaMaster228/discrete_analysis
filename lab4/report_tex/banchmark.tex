\section{Тест производительности}
{\itshape}

Я решил сравнить свою реализацию алгоритма Ахо-Корасик с наивным алгоритмом поиска. Тестирование происходило на тестах с 
миллионом слов. В первом тесте образец часто встречался в тексте, во втором - не встречался вообще.

\begin{alltt}
// Test 1
roma@DESKTOP-JD58QU2:~/Diskran/lab4$ ./a.out <test.txt >result.txt
Time: 0.164362 seconds
roma@DESKTOP-JD58QU2:~/Diskran/lab4$ ./naive <test.txt >result1.txt
Time: 0.397422 seconds
roma@DESKTOP-JD58QU2:~/Diskran/lab4$ diff result.txt result1.txt

// Test 2
roma@DESKTOP-JD58QU2:~/Diskran/lab4$ ./a.out <test.txt >result.txt
Time: 0.0644277 seconds
roma@DESKTOP-JD58QU2:~/Diskran/lab4$ ./naive <test.txt >result1.txt
Time: 0.148165 seconds
roma@DESKTOP-JD58QU2:~/Diskran/lab4$ diff result.txt result1.txt

\end{alltt}

По результатам тестирования видно, что алгоритм Ахо-Корасик значительно выиграл по времени у наивного алгоритма благодаря 
использованию префиксного дерева и суффиксных ссылок.
\pagebreak

