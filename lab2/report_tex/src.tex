\section{Описание}

B-дерево - это сильноветвящееся и идеально сбалансированное дерево поиска. Согласно \cite{Kormen} оно обладает следующими
свойствами:

\begin{enumerate} 
	\item Каждый узел дерева содержит следующие атрибуты:
	\begin{enumerate} 
		\item массив из $n$ ключей, упорядоченных по возрастанию
		\item массив из $n + 1$ указателей на дочерние узлы
	\end{enumerate} 
	\item Ключи узла дерева разделяют поддиапазоны ключей, хранящихся в поддеревьях: в $i$-ом дочернем узле хранятся элементы, 
	ключи которых больше $(i - 1)$-ого, но меньше $i$-ого ключа родительской вершины.
	\item Все листья расположены на одинаковой глубине.
	\item Каждый узел дерева должен содержать как минимум $t - 1$ ключ (исключение: корень должен содержать минимум 1 ключ). 
	Количество ключей не должно превышать $2 * t - 1$, где $t >= 2$ - степень дерева.
\end{enumerate}

Алгоритм поиска по B-дереву напоминает поиск в бинарном дереве. Если искомая вершина отсутствует в текущей вершине, 
то необходимо найти поддиапазон, которому она принадлежит, и продолжить поиск в соответствующем дочернем узле.
Вставка новых элементов осуществляется только в листовые узлы В-дерева. Место для вставки определяется алгоритмом поиска. 
Если в найденной вершине меньше $2 * t - 1$ элементов, то необходимо просто вставить новый ключ в массив так, чтобы
сохранилась его упорядоченность. В противном случае необходимо выполнить разбиение вершины: средний ключ перемещается 
в родительскую вершину, а первые и последние $t - 1$ ключей становятся его левым и правым ребенком соответственно.

При удалении элемента из В-дерева могут возникнуть следующие ситуации:

\begin{enumerate} 
	\item Удаляемый элемент находится в листовом узле, размер которого больше $t - 1$. В этом случае необходимо просто 
	удалить нужный ключ из листа.
	\item Удаляемый элемент находится во внутреннем узле. Если размер левого поддерева для удаляемого элемента больше $t - 1$, 
	то необходимо заменить этот элемент на максимальный элемент левого поддерева (расположен в листе) и удалить соответствующий 
	элемент из листа. Если размер правого поддерева больше $t - 1$, то необходимо проделать аналогичную операцию с минимальным
	элементом из правого поддерева. В противном случае необходимо выполнить слияние левого поддерева, удаляемого элемента 
	и правого поддерева в одну вершину и продолжить удаление из новой вершины.
	\item Если текущая вершина не содержит удаляемого элемента, то необходимо найти дочерний узел, в котором должна располагаться 
	вершина. Если размер найденного узла равен $t - 1$, то необходимо переместить один ключ из его родителя в найденный узел, 
	а крайний ключ из брата - в родителя. Если же размер обоих братьев этого узла тоже равен $t - 1$, то необходимо выполнить 
	слияние с одним из двух братьев и продолжить удаление.
\end{enumerate}

Для сериализации дерева в файл записывались структуры следующего вида: лист/не лист + количество элементов в вершине + 
[размер ключа + ключ + значение] для каждого элемента + аналогичные структуры для всех потомков, если текущая вершина не является листом.
При десериализации данные из файла считывались согласно схеме выше и копировались в В-дерево.

\pagebreak

\section{Исходный код}

Для хранения пар \enquote{ключ-значение} будем использовать структуру $TPair$, так как это удобно. Ключ будем хранить
в статическом массиве на 257 элементов типа char. А значения будем хранить как uint\_64. 
Перегрузим операторы сравнения для структуры $TPair$. Пара \enquote{ключ-значение} меньше, если её ключ лексикографически меньше.
Создадим структуру $TNode$ для хранения узла дерева. Структура состоит из массива ключей $TPair$, массива дочерних узлов $TNode*$, целочисленной переменной $keys\_num$, показывающей количество ключей в вершине в текущий момент, и булевой переменной $is\_leaf$, 
показывающей, является ли данная вершина листом. Переменная $DEGREE$ играет роль $t$ в B-дереве.

\begin{lstlisting}[language=C++]
const int DEGREE = 5;

struct TPair {
	char key[257];
	uint64_t value;
};

class TNode {
	public:
	bool is_leaf = true;
	int keys_num = 0;
	TPair keys[2 * DEGREE];
	TNode* children[2 * DEGREE + 1];
	
	TNode() {
		for (int i = 0; i < 2 * DEGREE + 1; ++i) {
			children[i] = nullptr;
		}
	}
};
\end{lstlisting}

Для хранения самого дерева создадим класс $TBTree$, который включает в себя указатель на корень, публичные методы для поиска, 
вставки, удаления элементов, сериализации и десериализации, а также приватный метод для рекурсивного удаления дерева. Эти методы
будут использовать другие функции, чтобы код удобнее читался. Все функции реализованы в соответсвии с описанием. 

\begin{lstlisting}[language=C++]
	
void SearchNode(TNode* node, char* str, TNode*& res, int& pos);  // if key not in tree, res will be nullptr
void SplitChild(TNode* parent, int pos); // splitting parent at pos
void InsertNode(TNode* node, TPair& KV);
int SearchInNode(TNode* node, char* str); // searching str. pos is index of str or index of a child with str
void RemoveNode(TNode* node, char* str);
void RemoveFromNode(TNode* node, int pos); // remove key[pos] in current node by making shifts
void MergeNodes(TNode* parent, int pos); // merging left child, parent[pos] and right child to left child
void Rebalance(TNode* node, int& pos); // rebalancing tree if a node has critical size
void NodeToFile(TNode* node, ostream& file);
void FileToTree(TNode* node, istream& file);
void DeleteTree(TNode* node);
	

class TBTree {
	public:
	TNode* root;
	
	TBTree() {
		root = new TNode;
	}
	
	~TBTree() {
		Delete();
	}
	
	void Search(char* str) const {
		TNode* res;
		int pos;
		SearchNode(root, str, res, pos);
		if (res == nullptr) {
			cout << "NoSuchWord\n";
		}
		else {
			cout << "OK: " << res->keys[pos].value << "\n";
		}
	}
	
	void Insert(TPair& KV) {
		TNode* search_res;
		int search_pos;
		SearchNode(root, KV.key, search_res, search_pos);
		if (search_res != nullptr) {
			cout << "Exist\n";
			return;
		}
		// if a root is full, we'll make a split
		if (root->keys_num == 2 * DEGREE - 1) {
			TNode* new_root = new TNode;
			new_root->is_leaf = false;
			new_root->children[0] = root;
			root = new_root;
			SplitChild(root, 0);
		}
		InsertNode(root, KV);
		cout << "OK\n";
	}
	
	void Remove(char* str) {
		TNode* res;
		int pos;
		SearchNode(root, str, res, pos);
		if (res == nullptr) {
			cout << "NoSuchWord\n";
		}
		else {
			RemoveNode(root, str);
			cout << "OK\n";
			// if our root is empty, its child becomes a new one
			if (root->keys_num == 0 && !root->is_leaf) {
				TNode* tmp = root->children[0];
				delete root;
				root = tmp;
			}
		}
	}
	
	void Serialize(ostream& out) {
		NodeToFile(root, out);
		cout << "OK\n";
	}
	
	void Deserialize(istream& in) {
		FileToTree(root, in);
		cout << "OK\n";
	}
	
	private:
	void Delete() {
		DeleteTree(root);
	}
};
}

	
\end{lstlisting}

\pagebreak

\section{Консоль}
\begin{alltt}
roma@DESKTOP-JD58QU2:~/Diskran/lab2$ make
g++ -pedantic -Wall -std=c++11 -Werror -Wno-sign-compare -O3 -lm main.cpp -o
solution
roma@DESKTOP-JD58QU2:~/Diskran/lab2$ ./solution
+ a 1
OK
+ b 2
OK
+ c 3
OK
+ d 4
OK
a
OK: 1
b
OK: 2
c
OK: 3
d
OK: 4
! Save test
OK
-a
OK
-b
OK
-c
OK
-d
OK
a
NoSuchWord
b
NoSuchWord
c
NoSuchWord
d
NoSuchWord
+ e 5
6
OK
! Load test
OK
a
OK: 1
b
OK: 2
c
OK: 3
d
OK: 4
e
NoSuchWord
\end{alltt}
\pagebreak

