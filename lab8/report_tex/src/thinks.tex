\section{Выводы}

Жадный алгоритм - это довольно простой и действенный метод решения задач на оптимизацию, который может быть полезен там, где не справляется динамическое программирование. Однако не все задачи могут быть решены с использованием жадных алгоритмов (типичный пример - задача о дискретном рюкзаке).

Жадные алгоритмы часто используются на практике. Но так как в реальном мире приходится работать с данными огромного размера, то вычислительных мощностей для точного алгоритма может не хватать. Поэтому применяются приближенные жадные алгоритмы, которые работают гораздо быстрее, но дают приблизительный ответ вместо точного.
\pagebreak
